\documentclass[12pt,a4paper,oneside]{article}
% nagyon sok kép esetén meggyorsítható a fordítás a draft móddal
% \documentclass[12pt,a4paper,oneside,draft]{report}
% ekkor a képek nem renderelődnek ki, csak placeholder lesz mérethelyesen
\usepackage[utf8]{inputenc} % mindenképp maradjon az utf-8 kódolás
\usepackage[magyar]{babel}
\usepackage[T1]{fontenc}
\usepackage{amsmath}
\usepackage{amsfonts}
\usepackage{amssymb}
\usepackage{graphics} % grafikus elemek, képek berakásához
%\usepackage{epsfig} % eps importáláshoz
%\usepackage{listings}
%\usepackage{sectsty}
%\usepackage{enumerate}
\usepackage{siunitx} % ezzel lehet hivatalosan megformázni: szám + mértékegység
%\usepackage{lastpage}
\usepackage{setspace}
\usepackage{hyperref} % PDF hivatkozásokhoz kell
\usepackage[hang]{caption}
%\usepackage{titling} % a title, author parancsok szabad használatához
\usepackage{xcolor}
\usepackage{multicol}
\usepackage[hyphenbreaks]{breakurl}
% A legendás siunitx package, ami a mértékegységek leírásának
% leghivatalosabb módja, ezeket a makrókat definiálja:

% használat: \SI[per-mode=symbol]{12345,678}{\kilo\amper\per\kandela\tothe{9}}
% ilyesmi lesz, csak jobb: 12345.67\, kAcd^{-9}

% egyszerűbb eset: \SI{50}{\ohm}


%% SI alapegységek:
% amper         -   \ampere
% kandela       -   \candela
% kelvin        -   \kelvin
% kilogramm     -   \kilogram
% méter         -   \metre  (!!)
% mól           -   \mole
% másodperc     -   \second

%% Származtatott SI egységek:
% becquerel     -   \becquerel
% celziusz fok  -   \degreeCelsius
% coulomb       -   \coulomb
% farád         -   \farad
% gray          -   \gray
% hertz         -   \hertz
% henry         -   \henry
% joule         -   \joule
% katal         -   \katal
% lumen         -   \lumen
% lux           -   \lux
% newton        -   \newton
% ohm           -   \ohm
% pascal        -   \pascal
% radián        -   \radian
% siemens       -   \siemens
% sievert       -   \sievert
% szteradián    -   \steradian
% tesla         -   \tesla
% volt          -   \volt
% watt          -   \watt
% weber         -   \weber

%% Elfogadott, nem SI mértékegységek
% nap           -   \day
% fok           -   \degree
% hektár        -   \hectare
% óra           -   \hour
% liter         -   \litre ( vagy \liter, de az előző kis "l" lesz, az utóbbi nagy "L")
% fokperc       -   \arcminute
% perc          -   \minute
% fokmásodperc  -   \arcsecond
% tonna         -   \tonne

%% Kísérleti úton megállapítható nem SI egységek:
% csillagászati egység      -   \astronomicalunit
% atomi tömegegység         -   \atomicmassunit
% bohr                      -   \bohr
% fénysebesség              -   \clight
% dalton                    -   \dalton
% elektrontömeg             -   \electronmass
% elektronvolt              -   \electronvolt
% egységtöltés              -   \elementarycharge
% hartree                   -   \hartree
% redukált planck állandó   -   \planckbar

%% Más nem SI egységek:
% ångström          -   \angstrom
% bár               -   \bar
% barn              -   \barn
% bel               -   \bel
% decibel           -   \decibel
% csomó             -   \knot
% higanymilliméter  -   \mmHg
% tengeri csomó     -   \nauticalmile
% neper             -   \neper

%% Prefixek:
% yocto-        -   \yocto  (-24)
% zepto-        -   \zepto  (-21)
% atto-         -   \atto   (-18)
% femto-        -   \femto  (-15)
% piko-         -   \pico   (-12)
% nano-         -   \nano   (-9)
% mikro-        -   \yocto  (-6)
% milli         -   \milli  (-3)
% centi-        -   \centi  (-2)
% deci-         -   \deci   (-1)

% deka-         -   \deca   (1)
% hekto-        -   \hecto  (2)
% kilo-         -   \kilo   (3)
% mega-         -   \mega   (6)
% giga-         -   \giga   (9)
% tera-         -   \tera   (12)
% peta-         -   \peta   (15)
% exa-          -   \exa    (18)
% zetta-        -   \zetta  (21)
% yotta-        -   \yotta  (24)


% a TikZ rajzoló modul, és a kapcsolási rajz készítő modul, ha kell
%\usepackage{tikz}
%\usepackage{circuitikz}

\pagenumbering{arabic}

\definecolor{rosewood}{rgb}{0.6, 0.0, 0.04}
\definecolor{indigo(dye)}{rgb}{0.0, 0.25, 0.42}

\usepackage[left=25mm,right=25mm,top=20mm,bottom=25mm]{geometry}\pagestyle{plain}

\singlespacing
%\onehalfspacing

\def\UrlBreaks{\do\/\do-}

\hypersetup
{
  	colorlinks,
  	citecolor=blue,
 	linkcolor=rosewood,
  	urlcolor=indigo(dye)
}
\title{Nem optikai elven működő \\ kvantum véletlenszám generátorok} 
\author{Szilágyi Gábor}
\date{\today}


\begin{document}
\maketitle
\section*{Bevezetés}
A véletlen számokat számos területen használnak. Ide tartozik a kriptográfia, ezen belül például az Internet Of Things (IoT) eszközök közötti kommunikáció, a digitális aláírások, a digitális szolgáltatások felhasználóinak azonosítása, az egyszerhasználatos kulcsok (One-Time Pad) generálása, vagy a banki tranzakciók adminisztrációja. Bizonyos szimulációs eljárásoknál -- elsősorban a Monte-Carlo módszernél -- a szimulációs eredmények helyessége a felhasznált véletlen számok véletlenszerűségén nyugszik. Választások lebonyolításánál és szerencsejátékoknál is fontos a felhasznált véletlen számok kompromittálatlansága, itt elsősorban sorsolásra használják fel a véletlen számokat. Mesterséges intelligenciák evolúciójánál a túlélő példányok kiválasztása, tulajdonságaik összekeverése, valamint a mutációik is fontos, hogy teljesen véletlenszerűek legyenek. Neurális hálóknál az élek véletlenszerű súlyozása is véletlen számokat igényel. Végül, de nem utolsósorban a kriptovalutákat kezelő infrastruktúrákban is fontos szerepe van a véletlen számoknak.
\par
A fenti felsorolás messze nem kimerítő, ebből is látszik, hogy a véletlen számok nagy hatással vannak a mai világ működésére. A felhasznált véletlen számok megjósolhatóságának súlyos következményei lehetnek, mert sok esetben teljesen meghiúsítják az őket felhasználó alkalmazások működését, ami nagy veszteségeket okozhat. Ilyen esetekre néhány példa a közelmúltból \cite{altalanos-random}:
\begin{itemize}
	\item 2010-ben a Sony PlayStation 3 játékkonzol 77 millió felhasználójának adatait ellpoták a támadók. A támadáshoz a cég ECDSA (Elliptic Curve Digital Signatura Algorithm) implementációjának hibáját használták ki, ami miatt ugyanazt a számot többször használta fel az algoritmus a felhasználók azonosításához \cite{sony}.
	\item 2012-ban két kutatócsoport több olyan RSA titkosítási kulcsot hozott napvilágra, amiket addig aktívan használtak az interneten biztonságosnak hitt kulcsként, de feltörhetőek voltak amiatt, hogy nem megfelelő véletlenszám generátort használtak a kulcsok generálásakor \cite{bad-rsa}.
	\item 2015-ben 18866 Bitcoint loptak el a Bitstamp kriptovaluta tőzsdén, ami az ott forgalomban lévő kriptopénz 12\%-át jelentette ekkor. A támadás kapcsán a véletlen számok hibás generálásának kihasználására utaló nyomok kerültek napvilágra \cite{bitstamp}.
\end{itemize}
\section*{A véletlenszerűség definiálása}
Ugyan elég jól körül lehet írni a valódi véletlent és fel tudunk sorolni olyan tulajdonságokat, amik biztosan igazak egy véletlenszerűen előállított bitsorozatra, de a matematika mai eszköztárával nem sikerült általános, formális definíciót találni a valódi véletlenre.
\par
A valódi véletlenszerűség eldöntéséhez a legjobb, amit tehetünk, az a különböző tesztek lefuttatása a vizsgált bitsorozaton. Ezek semmilyen véges hosszú bitsorozatra alkalmazva nem tudnak egyértelmű választ adni arra a kérdésre, hogy az valóban véletlen-e, de a segítségükkel legalább azt el lehet dönteni, hogy a generálás módszere a vizsgált szempontokból mennyire áll közel az igazi véletlenhez.
\par
Matematikailag nehezen kezelhető a tökéltes véletlenszerűség, de vannak olyan fizikai folyamatok, amelyek a tudomány mai állása szerint teljesen kiszámíthatatlanok, emiatt megfelelnek a tökéletes véletlenszerűségről alkotott elképzeléseknek. Az ilyen fizikai folyamatok a kvantumfizika területén keresendőek, tehát kis méretskálán működnek, nagyságrendileg az egyedüli részecske (általában atom, elektron vagy foton) szintjén. Több ilyen folyamat is ismert, a rájuk épülő véletlenszám-generátorok kivitelezhetősége változó. Az említett fizikai folyamatok jó része optikai alapú, vagyis fotonok véletlenszerű viselkedésére épül, emiatt alapvetően két csoportba szokták osztani: optikai- és nem optikai folyamatokra. Ez a tanulmány az utóbbi csoportba tartozó folyamatokra építő véletlenszám generátorokkal foglalkozik.
\par
A kvantum folyamatok kimenetele lehet teljesen kiszámíthatatlan, de ha egy ilyen folyamatot felhasználó véletlenszám generátort realizálunk, akkor az valamilyen mértékben ,,szennyezni'' fogja a kvantum folyamatra vonatkozó mérési eredményeket valamilyen determinisztikus zajjal. Emiatt nyer értelmet a véletlenszám generátorok kimenetének osztályozása aszerint, hogy mennyire véletlenszerűek.
\par
A véletlenszám generátorok minőségi osztályozására előre definiált tesztcsomagokat szokás használni, amelyek több szempontból ellenőrzik a generátor viselkedését. Egy egyszerűbb szempont az ismétlődő minták keresése a generált bitsorozatban. Két véletlenszám generátorok működésének ellenőrzésére használt tesztcsomag a NIST SP 800-22 NIST \cite{nist1} és a SP 800-90B \cite{nist2} (NIST -- National Institute of Science and Technology).
\section*{Véletlenszám generátorok}
Egy véletlenszám generátornak két paramétere kritikus, az egyik a számok megjósolhatatlanságának minősége, a másik a generálás bitsebessége.
\par
Bizonyos alkalmazásoknál a két paraméter közül nagy bitsebesség sokkal fontosabb, ekkor megengedhető lehet olyan véletlenszám generátor használata, ami valójában determinisztikus módon hozza létre a kimeneti bitsorozatát, ezek a pszeudo-véletlen generátorok, amelyek gyakran (de nem feltétlenül) szoftveresen vannak implementálva. A pszeudo-véletlenszám generátorok hátránya, hogy a kimeneti bitsorozatuk különböző módokon előre kiszámítható lehet, emiatt nem felelnek meg a tökéletes véletlenszerűség kritériumainak.
\par
Ahol a generált bitsorozat ,,minősége'' fontos, ott olyan fizikai folyamatot érdemes alapul venni a bitek előállításához, ami valóban kiszámíthatatlan. Az ilyen folyamatok a klasszikus fizika helyett a kvantumfizika területén keresendőek. A kvantum folyamatok használatának egyik előnye, hogy a véletlen számokkal szemben támasztott követelményeknek automatikusan megfelelnek. A kvantumszámítógépek fejlesztése az elmúlt években olyan szintre jutott, hogy a determinisztikus úton előállított véletlen bitsorozatokat már az a veszély fenyegeti, hogy még ha ugyan klasszikus számítógépekkel nem is, de kvantumszámítógépek segítségével megjósolhatóak lehetnek. Abban rejlik a kvantum véletlenszám generátorok másik előnye, hogy a segítségükkel előállított bitsorozatok elméletileg kvantumszámítógépek segítségével is megjósolhatatlanok.
\par
Létezik a fent leírt végletek közötti kompromisszumos megoldás, ami abban áll, hogy egy relatíve lassú kvantum véletlenszám generátor (entrópiaforrás -- entropy source) kimenetét egy determinisztikus algoritmusban kiindulási változóként (random seed) használjuk. A determinisztikus algoritmus a saját kimenetén egy a bemenetihez képest sokszoros hosszú bitsorozatot generál, de ez az új bitsorozat az algoritmus paraméterei és a kiindulási változó ismeretében teljesen reprodukálható. Ezért ezzel a megoldással nem minden esetben lehet kiváltani a teljesen kvantum generátort.
\par
A fenti okok miatt egyre nő és egyre több alkalmazási területet érint a kvantum folyamatokon alapuló véletlenszám generátorok iránti igény.
\section*{Radioaktív bomlás alapú generátorok}
A  radioaktív bomlási folyamatok instabil atommagokban játszódnak le, amik úgy veszítenek energiát a folyamat során, hogy valamilyen részecskét (részecskéket) bocsátanak ki és más össztételű atommaggá alakulnak közben. Ezeknek a bomlási folyamatoknak a bekövetkezési ideje az előre teljesen meghatározhatatlan és kívülről befolyásolhatatlan. Ezek a folyamatok a radioaktív izotóp felezési idejével ($\lambda$) jellemezhetőek, ami azt az időtartamot jelenti, ami alatt pontosan 1/2 valószínűséggel következik be egy adott atommagban a bomlási folyamat. Ha makroszkopikus mennyiséget veszünk egy ilyen radioaktív izotópból, akkor abban az elbomlás nélkül megmaradt atomok száma exponenciálisan, gyakorlatilag folyamatosan csökken -- találóan felezési időnként feleződik.
\par
A radioaktív bomlások különböző kisugárzott részecskékkel járhatnak, a különböző bomlási folyamatok jellemzőek az egyes instabil izotópokra. A kibocsátott részecskék között lehetnek neutrínók, pozitronok, valamint hélium atommagnál nehezebb atommagok is, de a gyakoribb kisugárzott részecsketípusok az alfa-, béta- és gamma részecskék.
\begin{itemize}
	\item Az alfa részecske egy hélium atommag. Az alfa-sugárzás relatíve könnyen árnyékolható, erre egy papírlap is képes. A veszélye abban rejlik, hogy jelentős a roncsoló hatása, ezért veszélyes az élőlényekre.
	\item A béta részecske egy elektron. Megkülönböztetnek $\beta^-$ és $\beta^+$ sugárzást, ezek elektron és pozitron (antielektron) sugárzást jelentenek sorrendben. A béta sugárzás valamivel nehezebben árnyékolható, mint az alfa, de még mindig nem túl nehezen -- például egy vékony alumíniumlemez is jó árnyékoló hatású.
	\item A gamma részecske egy nagyenergiájú foton, ami nehezen árnyékolható, de emellett nagy roncsoló hatású is, ezek kombinációja kifjezetten veszélyessé és nehezen kezelhetővé teszi. A gamma-sugárzás elleni árnyékolás körülményes, például egy vastag ólomréteg tud hatásos lenni ebben a tekintetben.
\end{itemize}
\par
A
\section*{Alagúteffektus alapú generátorok}

Bibliography
\\\\
\begin{thebibliography}{12}
	\sloppy
	\bibitem{altalanos-random} Nature Scientific Reports -- Quantum generators of random numbers \\ \url{https://www.nature.com/articles/s41598-021-95388-7}
	\bibitem{sony} fail0verflow. Console Hacking 2010, PS3 Epic Fail (2011) \\ \url{https://events.ccc.de/congress/2010/Fahrplan/attachments/1780_27c3_console_hacking_2010.pdf}
	\bibitem{bad-rsa} Lenstra, A. K., Hughes, J. P., Augier, M., Kleinjung, T. \& Wachter, C. Ron was wrong, whit is right (2012) \\ \url{https://eprint.iacr.org/2012/064.pdf}
	\bibitem{bitstamp} CNN. Did a Bitcoin Exchange Just Lose 12\% of Its Bitcoins? Possible Bitstamp Hack Address Contains 18,866 Stolen BTC (2015) \\ \url{https://www.ccn.com/bitcoin-exchange-just-lose-12-bitcoins-possible-bitstamp-hack-address-contains-18866-stolen-btc/}
	\bibitem{nist1} NIST -- A Statistical Test Suite for Random and Pseudorandom Number Generators for Cryptographic Applications \\ \url{https://nvlpubs.nist.gov/nistpubs/Legacy/SP/nistspecialpublication800-22r1a.pdf}
	\bibitem{nist2} NIST -- Recommendation for the Entropy Sources Used for Random Bit Generation \\ \url{https://nvlpubs.nist.gov/nistpubs/SpecialPublications/NIST.SP.800-90B.pdf}
\end{thebibliography}
\end{document}